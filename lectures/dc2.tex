\documentclass[11pt,twoside]{article}
\usepackage{amsmath,amssymb,graphicx}
\usepackage{latexsym}

\newcommand{\coursenumber}{CIS 121}
\newcommand{\coursetitle}{Data Structures and Algorithms}
\newcommand{\docdate}{February 02, 2016}
\newcommand{\duedate}{February 02, 2016}
\newcommand{\doctitle}{Running time, Divide and Conquer}
\newcommand{\student}{PUT YOUR NAME HERE}

\pagestyle{myheadings}
\markboth{\hfill\doctitle\hfill\docdate}{\docdate\hfill\doctitle\hfill}

\addtolength{\textwidth}{1.00in}
\addtolength{\textheight}{1.00in}
\addtolength{\evensidemargin}{-0.5in}
\addtolength{\oddsidemargin}{-0.5in}
\addtolength{\topmargin}{-.25in}
%\setlength{\footskip}{0pt}

\newcommand{\E}{{\mbox {\bf E}}}
\newcommand{\Var}{{\mbox {Var}}}

\newcommand{\solution}{\bigskip\hrule\bigskip}
\newcommand{\problembreak}{\bigskip\hrule\bigskip}

\def\ni{\noindent}

\renewcommand{\theenumi}{\textbf{\Alph{enumi}}}

\newcommand{\comment}{$\rhd$\ \ }

\begin{document}

\thispagestyle{empty}

\begin{center}
\Large\bf\coursetitle\\[2pt]\doctitle\\ \large\docdate
% \\[2pt]\student       % Uncomment this line to insert your name
\end{center}
\vspace*{0.10in}
%{\noindent{\bf Given:}\ \ \docdate}
%{\hfill{\bf Due:}\ \ \duedate}

\hrule
\section*{Quicksort}
In quick sort, we first decide on the pivot. This could be the element
at any
location in the input array. The function \texttt{Partition} is then
invoked. \texttt{Partition} accomplishes the following: it places the
pivot in the location that it should be in the output and places all
elements that are at most the pivot to the left of the pivot and all
elements greater than the pivot to its right. Then we recurse on both
parts. The pseudocode for Quick Sort is as follows. 
\begin{verbatim}
  QSort(A[lo..hi])
    if hi <= lo then
      return
    else
      pivotIndex = floor((lo+hi)/2)  (this could have been any location)
      loc = Partition(A, lo, hi, pIndex)
      QSort(A[lo..loc-1])
      QSort(A[loc+1..hi])     
\end{verbatim} 

One possible implementation of the function \texttt{Partition} is as follows.
\begin{verbatim}
  Partition(A, lo, hi, pIndex)
    pivot = A[pIndex]
    swap(A, pivotIndex, hi)
    left = lo
    right = hi-1
    while left <= right do
       if (A[left] <= pivot) then
          left = left + 1
       else
          swap(A, left, right)
          right = right - 1
    swap(A, left, hi)
    return left
\end{verbatim}

The worst case running time of the algorithm is given by
\[
T(n) = \left\{ \begin{array}{ll}
       1, &  n=1\\
       T(n-1) + cn, &  n \geq 2 
      \end{array} \right.
\]
Hence the worst case running time of \texttt{QSort} is $\Theta(n^2)$.

An instance where the quick sort algorithm in which the pivot is
always the first element in the input array performs poorly is an
array in descending order of its elements.
 
\section*{Randomized Quick Sort}
In randomized version of quick sort, we pick a pivot uniformly at
random from all possibilities. We will
now show that the expected number of comparisons made in randomized
quick sort equals $2n\ln n + O(n)$.

\paragraph{Theorem.} For any input, the expected number of comparisons
made by randomized quick sort is $2n\ln n +O(n)$.

\paragraph{Proof.} Let $y_1,y_2, \ldots, y_n$ be the input elements in
sorted order. Let $X$ be a random variable denoting the total
number of pair-wise comparisons made between elements of the input
array $A$. Let $X_{ij}$ be a random variable denoting the total number
of times elements $y_i$ and $y_j$ are compared during the
algorithm. Observe the following.
\begin{itemize}
\item
comparisons between elements in the input array are done only in the
function \texttt{Partition}.
\item
there are $n-1$ distinct pivots chosen in the algorithm and hence
$n-1$ calls to \texttt{Partition}. 
\item
two elements are compared only if one of them is a pivot.
\end{itemize} 
Let $X_{ij}^{k}$ be a random variable that is 1, if elements $y_i$ and
$y_j$ are compared in the $k$th call to \texttt{Partition}. Note that 
\[
X = \sum_{i=1}^{n-1}\sum_{j=i+1}^{n}X_{ij} \mbox{~~~~~and~~~~~} X_{ij} = \sum_{k=1}^{n-1}X_{ij}^k
\]
By the linearity of expectation, we have 
\begin{equation}
\label{eqn:Ex}
\E[X] = \sum_{i=1}^{n-1}\sum_{j=i+1}^{n}\E[X_{ij}]
\end{equation}
We will now calculate $\E[X_{ij}]$. By the linearity of expectation,
we have
\[
\E[X_{ij}] = \sum_{k=1}^{n-1}\E[X_{ij}^{k}] = \E[X] = \sum_{k=1}^{n-1}\Pr[X_{ij}^{k}=1]
\]
Let $z$ be the first call to \texttt{Partition} during which one of
the elements from $y_i,y_{i+1}, \ldots, y_n$ is used as the
pivot. From our observations above, note that for all $h$ and all $\ell$
such that  $1\leq h < z$ and $z< \ell <n$, $X_{ij}^h=0$ and
$X_{ij}^{\ell}=0$. If one of $y_i$ or $y_j$ is chosen as the $k^{th}$
pivot then clearly, $X_{ij}^{k}=1$, otherwise, $X_{ij}^{k}=0$, and
$y_i$ and $y_j$ will be separated into different sublists and hence will
never be compared again.  Thus we have
\begin{equation}
\label{eqn:Exij}
\E[X_{ij}] = \Pr[X_{ij}^{k}=1] = \frac{2}{j-i+1}
\end{equation}
Combining equations (\ref{eqn:Ex}) and (\ref{eqn:Exij}), we get
\begin{align*}
\E[X] &= \sum_{i=1}^{n-1}\sum_{j=i+1}^{n}\E[X_{ij}] \\
      &= \sum_{i=1}^{n-1}\sum_{j=i+1}^{n}\frac{2}{j-i+1} \\
      &= \sum_{k=2}^{n}\frac{2(n-k+1)}{k} \\
      &= (n+1)\sum_{k=2}^{n}\frac{2}{k} - 2(n-1)  \\
      &= 2(n+1)\sum_{k=1}^{n}\frac{1}{k} - 2(n-1)-2(n+1) \\ 
      &= 2(n+1)\ln n + c - 4n  \mbox{~~~~~~~~~~~~~~($c$ is a constant less than 1)}\\
      &= 2n\ln n + \Theta(n)
\end{align*}


\paragraph{Example.} Find the running time expressed by the following
recurrence. Assume that $n$ is a power of 3.
\[
T(n) = \left\{ \begin{array}{ll}
      5T(n/3) + n^2, &  n \geq 2 \\
      1, & \mbox{n=1}
      \end{array} \right.
\]
\begin{align*}
T(n) & =  5T(n/3) + n^2 \\
& =  5^2T(n/3^2) + 5(n/3)^2 + n^2 \\
& = 5^3T(n/3^3) + 5^2(n/3^2)^2 + 5n^2/3^2 + n^2 \\
&  \ldots \\
&  \ldots \\
& = 5^kT(n/3^k) + n^2\left (1 + \frac{5}{9} + \left (\frac{5}{9}\right
)^2 +\ldots + \left(\frac{5}{9} \right )^{k-1}\right ) \\ 
\end{align*}
\noindent
The recursion bottoms out when $n/3^k = 1$, i.e., $k=\log_3 n$. Thus, we get
\begin{align*}
T(n) & =  5^{\log_3n}T(1) + n^2\sum_{i=0}^{k-1}(5/9)^i \\
     & = 5^{\log_3n} + n^2\left (\frac{(5/9)^{\log_3 n}-1}{(5/9) -1} \right ) \\
     & = 5^{\log_5 n/\log_5 3} + \frac{9n^2}{4}\left (1 - (n^{\log_35}/n^2) \right ) \\
     & = n^{\log_35} + \frac{9n^2}{4} - \frac{9n^{\log_35}}{4} \\
     & = \frac{9n^2}{4} - \frac{5n^{\log_35}}{4} \\
     & = \Theta(n^2)
\end{align*}

\paragraph{Example.} Solve the following recurrence. Assume that $n$
is a power of 2.
\[
T(n) = \left\{ \begin{array}{ll}
      3T(n/2) + n, &  n \geq 2 \\
      1, & \mbox{n=1}
      \end{array} \right.
\]
\paragraph{Solution.}
\begin{align*}
T(n) & =  3T(n/2) + n \\
& =  3^2T(n/2^2) + 3n/2 + n \\ 
& = 3^3T(n/2^3) + (3/2)^2n + 3n/2 + n \\ 
&  \ldots \\
&  \ldots \\
& = 3^kT(n/2^k) + n\sum_{i=0}^{k-1}(3/2)^i 
\end{align*}
\noindent
The recursion bottoms out when $n/2^k = 1$, i.e., $k=\lg n$. Thus, we get
\begin{align*}
T(n) & =  3^{\log_2n}T(1) + n\sum_{i=0}^{\lg n-1}(3/2)^i \\
     & = 3^{\log_2n} + n\left (\frac{(3/2)^{\lg n}-1}{3/2 -1}\right ) \\
     & = n^{\frac{\log_3n}{\log_32}} + 2n((3/2)^{\log_{3/2}n/\log_{3/2}2}-1)\\
     & = n^{\log_23} + 2n(n^{\lg (3/2)}-1) \\
     & = n^{\log_23} + 2n(n^{\lg 3- \lg 2)}-1) \\
     & = n^{\log_23} + 2n(n^{\lg 3-1}) - 2n \\
     & = n^{\log_23} + 2n^{\lg 3} -2n\\
     & = 3n^{\log_23} -2n\\
     & = \Theta(n^{\lg 3}) ~~~~~~ (\mbox{can be argued by setting $c=3$ and $n_0=1$})  
\end{align*}

\end{document}
