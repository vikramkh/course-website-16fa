\documentclass[11pt,twoside]{article}
\usepackage{amsmath,amssymb,graphicx}
\usepackage{latexsym}

\newcommand{\coursenumber}{CIS 121}
\newcommand{\coursetitle}{Data Structures and Algorithms}
\newcommand{\docdate}{February 09, 2016}
\newcommand{\duedate}{February 09, 2016}
\newcommand{\doctitle}{Lecture Outline}
\newcommand{\student}{PUT YOUR NAME HERE}

\pagestyle{myheadings}
\markboth{\hfill\doctitle\hfill\docdate}{\docdate\hfill\doctitle\hfill}

\addtolength{\textwidth}{1.00in}
\addtolength{\textheight}{1.00in}
\addtolength{\evensidemargin}{-0.5in}
\addtolength{\oddsidemargin}{-0.5in}
\addtolength{\topmargin}{-.25in}
%\setlength{\footskip}{0pt}

\newcommand{\solution}{\bigskip\hrule\bigskip}
\newcommand{\problembreak}{\bigskip\hrule\bigskip}

\def\ni{\noindent}

\renewcommand{\theenumi}{\textbf{\Alph{enumi}}}

\newcommand{\comment}{$\rhd$\ \ }

\begin{document}

\thispagestyle{empty}

\begin{center}
\Large\bf\coursetitle\\[2pt]\doctitle\\ \large\docdate
% \\[2pt]\student       % Uncomment this line to insert your name
\end{center}
\vspace*{0.10in}
%{\noindent{\bf Given:}\ \ \docdate}
%{\hfill{\bf Due:}\ \ \duedate}

\hrule
~\\

\ni
\section*{The Stack ADT}
An \textit{abstract data type} (ADT) is an abstraction of a data
structure. It specifies the type of data stored and different
operations that can be performed on the data. It's like a Java
interface -- it specifies the name and definition of the methods, but
hides their implementations.

In the stack ADT, the data can be arbitrary objects and the main
operations are \texttt{push} and \texttt{pop} that allow insertions and
deletions in a last-in, first-out manner. One way to implement a stack
is to use an array. Note that an
array has a fixed size, so in a \textit{fixed capacity} implementation
of the stack, the number of items in the stack can be no more than the
size of the array. This implies that to use the stack one must
estimate the maximum size of the stack ahead of time.  To make the
stack have unlimited capacity, we will adjust dynamically the size of
the array so that it is both sufficiently large to store all of the
items and not so large so as to waste an excessive amount of
space. We will consider two strategies -- \textit{incremental
  strategy} in which the array capacity is increased by a constant
amount when the stack size equals the array capacity 
and the \textit{doubling strategy}, in which the
size of the array is doubled when the stack size equals the array
capacity. Since arrays cannot be dynamically
resized, we have to create a new array of increased size and copy the
elements from the old array to the new array. 

We will first analyze the incremental strategy. 

\begin{verbatim}
push(obj)
// s: stack size
// a: array capacity
// c: initial array size and also the increment in array size
A[s] = obj
s = s+1
if s == a then
  a = a+c
  copy contents of old array to the new array
\end{verbatim}


Let $c$ be the intial
size of the array and let $c$ be the amount by which the array size is
increased during each expansion. Consider a sequence of $n$ push
operations. Note that after every $c$ push operations the array
expansion happens. The $n$ push operations cost $n$. The cost of the
first expansion is $c$ (since $c$ elements are being copied), the cost
of the second expansion is $2c$ (since $2c$ elements are being
copied), and so on. Thus the total cost $T(n)$ is given by
\[
T(n) = n + c + 2c+ 3c + \ldots + n = n + c(1+2+\ldots + n/c) =
n+\frac{c(n/c)(n/c+1)}{2} = \Theta(n^2)
\]

We will now analyze the doubling method (pseudo code given below). 

\begin{verbatim}
push(obj)
// s: stack size
// a: array capacity
A[s] = obj
s = s+1
if s == a then
  a = 2*a
  copy contents of old array to the new array
\end{verbatim}


We will use \textit{amortized analysis}
which means that we will be finding the time-averaged cost for a
sequence of operations. In other words, it is the time required to
perform a sequence of operations averaged over all the operations
performed. Let $T(n)$ be the total time needed to perform a series of $n$ push
operations. Then the amortized time of a push operation is $T(n)/n$.
Note that this is different from our usual notion of "average
case analysis" -- we're not making any assumptions about
inputs being chosen at random -- we're just averaging over
time. Note that the total
real cost of a sequence of operations will be bounded by the total
of the amortized costs of all the operations. 

Let $s$ denote the number of objects in the stack at any given time
and let $a$ be the array capacity at any given time. When $s < a$,
\texttt{push(obj)} 
is a constant time operation. However, when the stack is full, i.e.,
when $s=a$, we double 
the size of the array. Thus
the cost of \texttt{push(obj)} in this case is $O(s)$ as we have to
move $s$ items from the old array into the new array (the cost of
allocating and freeing the array is $O(s)$). The worst case cost of a
push operation is $O(n)$ and hence the cost of $n$ push operations if
$O(n\log n)$ (since there are $O(\log n)$ expansions). Is this tight?

If we start from an empty stack what is the cost of a sequence of $n$
push operations? As before, the cost of the $n$ push operations is
$n$. The cost of expansion (doubling) is at most $1+2+4+\ldots + n/2 +n <
2n$. Thus the total cost is at most $3n$ = $O(n)$. Thus the amortized cost of an
operation is 3, even though the worst case time complexity of a single
push operation is $O(n)$. 

Another way to analyze the doubling scheme is as follows: Each object
pays for the operation performed on it. When the array is doubled,
only the elements that have never been moved before pay for all the
elements that are being moved to the new array. Since the number of
such elements is exactly half the number of all elements in the stack,
each element pays a cost of 2 for the move. Thus the total cost
incurred by each element is 3 and hence $T(n) = O(n)$.

Similarly, in \texttt{pop()} after removing the object from the stack,
if the stack size is significantly less than the array capacity then
we resize the array. More specifically, when the stack size is equal to
one-fourth of the array size then we reduce the size of the array
to half its current capacity. After resizing, the array is still half
full and can accommodate a substantial number of push and pop
operations befiore having to resize the array. The pseudocode for the
pop operation is as follows.
\begin{verbatim}
pop()
item = A[s]
s = s-1
if (s < a/4) then
  a = a/2
\end{verbatim}

\section*{Queues}
A queue is a collection of objects that is based on a first-in,
first-out policy. The main operations in a Queue ADT are
\texttt{enqueue(obj)} and \texttt{dequeue}. The enqueue operation
inserts an element at the end of the queue. The dequeue operation
removes and returns the element at the front of the queue. Queues can
also be implemented using expandable arrays. However, unlike a stack,
in a queue we need to keep track of both the head and the tail of the
queue. Head of the queue points to the first element in the queue and
the tail points to the last element in the queue. Note that when we
dequeue an element, the element is removed from the head of the queue
and when an element is enqueued, an element is inserted at the tail of
the queue. When the tail points to the end of the array, it may not
mean that the array is full. This is because some elements may have
been popped off and the head of the queue may not be pointing to the
beginning of the array, meaning that there may be room at the
beginning of the array. To address this we use a wrap around
implementation. This way, we expand the array only when every slot in
the array contains an element, i.e., when the queue size equals the
array capacity. When copying the queue elements into the new array, we
``unwind'' the queue so that the head points to the beginning of the
array. 
 
\end{document}
